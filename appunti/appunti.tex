\hypertarget{ftp-file-transfer-protocol}{%
\subsubsection{FTP (File Transfer
Protocol)}\label{ftp-file-transfer-protocol}}

\begin{itemize}
\tightlist
\item
  Trasferimento file da/verso host remoto
\item
  Modello client/server

  \begin{itemize}
  \tightlist
  \item
    Client: chi inizia la richiesta di trasferimento (da/verso host
    remoto)
  \item
    Server: remote host
  \end{itemize}
\end{itemize}

Utilizza \textbf{due connessioni TCP parallele}: 1. \textbf{Control
connection}: invio di informazioni di controllo (username, password,
comandi change directory, comandi trasferimento file). Invio
informazioni in FTP \emph{out-of-band} (HTTP \emph{in-band},
nell'header). 2. \textbf{Data connection}: usata per il vero e proprio
invio dei file.

\textbf{Control connection} è persistente, la \textbf{Data connection}
si apre e si chiude ad ogni inizio e fine del trasferimento di un file.

\begin{center}\rule{0.5\linewidth}{\linethickness}\end{center}

\textbf{DNS} (Domain Name System) E' un servizio che \textbf{traduce} i
nomi degli host nei loro indirizzi IP usando un \textbf{Database
distribuito}. E' \emph{distribuito} perche': - unico punto di fallimento
- alto volume di traffico - database centralizzato \emph{lontano} -
manutenzione

Altri servizi: 1. \textbf{Host aliasing} 2. \textbf{Mail aliasing} 3.
\textbf{Load distribution} - usato per distribuire il carico tra server
replicati, nel caso di siti con tanto traffico per cui vengono usati
molteplici server (quindi ognuno con il \textbf{proprio indirizzo IP}
diverso da quello degli altri). Il Database DNS contiene tutti questi
indirizzi IP.

Il \textbf{DNS} è un database \textbf{distribuito e gerarchico} e ha tre
classi: 1. \emph{Root Server} 2. \emph{TLD (top-level-domain) server} -
.com, .org, .edu, ecc. 3. \emph{Server Autoritativi} - i server
\emph{DNS} propri di un'organizzazione che mappa i propri IP agli host
name

Esiste anche il \textbf{DNS server locale}, quello fornito dagli
\emph{ISP}. E' detto anche \emph{default name server}. Quando un host fa
una \emph{DNS query} essa è mandata al \emph{DNS server locale} che: -
ha una \textbf{cache} delle recenti ``traduzioni'' (ma può essere
vecchia). - fa da proxy e inoltra la query nella gerarchia.

\textbf{DNS Resolutions}: 1. \textbf{\emph{Iterativa}} - il server
contattato risponde con il nome del server da contattare 2.
\textbf{\emph{Ricorsiva}} - i server vengono contattati in modo
ricorsivo. Tanto lavoro nei livelli alti della gerarchia

\hypertarget{resource-record-record-di-risorsa}{%
\subsubsection{Resource Record (Record di
risorsa)}\label{resource-record-record-di-risorsa}}

Sono i record memorizzati nei database e hanno i seguenti campi:
\textbf{(\(Name, Value, Type, TTL\))}

\begin{itemize}
\tightlist
\item
  \(Type\) = \(A\) -\textgreater{} \textbf{Name} = hostname,
  \textbf{Value} = IP dell'hostname. Es: (relay1.bar.foo.com,
  145.37.93.126, A)
\item
  \(Type\) = \(NS\) -\textgreater{} \textbf{Name} = dominio,
  \textbf{Value} = hostname del \textbf{DNS server autoritativo} che sa
  come ottenere gli indirizzi IP dell'host del dominio. Es: (foo.com,
  dns.foo.com, NS)
\item
  \(Type\) = \(CNAME\) -\textgreater{} Es: (foo.com, relay1.bar.foo.com,
  CNAME)
\item
  \(Type\) = \(MX\) -\textgreater{} mail server. Es: (foo.com,
  mail.bar.foo.com, MX)
\end{itemize}

\textbf{PTR Resource Record} - e' un record particolare che mi permette
di ricevere l'\textbf{hostname} da un \textbf{indirizzo IP}. Questa
operazione viene chiamata \emph{Reverse DNS}. Viene spesso usato per
verificare l'affidabilita' di una macchina (in quanto i manager di rete
assegnano domain name solo alle macchine di cui hanno fiducia).

\begin{center}\rule{0.5\linewidth}{\linethickness}\end{center}

\hypertarget{file-distribution}{%
\subsubsection{File Distribution}\label{file-distribution}}

**Client-